
\chapter{Conclusion}
This research begins with a brief introduction and review of HCI, VR, Haptic, etc., in Chapter 1, gradually narrowing down the field to finally determine the research direction of the motion state of self-moving objects. In Chapter 2, we draw inspiration and analysis from precedents in various classified fields, fully leverage the advantages of haptics compared to other sensory content, and compare our research plan with external design plans to determine the design idea and application scenarios of the final design plan. In Chapter 3, after trying and comparing different types of haptic device prototypes, we finally decided on a haptic design plan mainly presented through a roller. The goal is to convey to users the research objectives of the motion state of self-moving objects through haptics. In Chapter 4, we verified through experiments the effectiveness of MotionPerformer in transmitting motion information to users, including three main contents: displacement, rotation, and speed. Furthermore, two experiments were conducted to explore and verify that MotionPerformer can strengthen the interaction between users and future AI-powered/intelligent automation systems, and enhance experiences, immersion, and the entertainment of the system in virtual space under entertainment or simulation scenarios by adding haptic feedback.

Overall, the research objectives of this study have been basically achieved, and all hypotheses have been established. MotionPerformer can indeed effectively deliver the motion information of self-moving objects to users, but theoretically, the design plan of this study can improve the accuracy of specific information transmission of the device more effectively by replacing more efficient and reliable components. Moreover, there are still places in the design that are worth optimizing, such as adjusting the placement of components, optimizing the handheld design to wearable design, the visual effects and interaction system in the virtual scene are not complete enough, and the VR usage environment that did not appear in the design. All of these are worth updating in subsequent versions. However, through a relatively simple experimental environment, we still proved that such a haptic design as MotionPerformer can enhance the interaction and connection between users and intelligent automation systems in the future, such as the driving scenario in this research. Users can indeed perceive the sense of agency of active driving in an inactive driving environment through MotionPerformer, which has a very positive role in considering how to interact with intelligent automation systems, such as drones and robots in the future. And most importantly, MotionPerformer has strong versatility. Its positive impact is not only manifested in automation interactions, it can be used in any virtual scene's user-avatar interaction process, and effectively enhance user immersion and interactivity. 

In summary, MotionPerformer can help users establish a force reference in virtual space or non-real operation environments, share motion information between the user and the interactive object, and we believe this will bring more experiences and possibilities to future human-computer interaction methods.